\newcommand\eq[1]{\Cref{eq:#1}\xspace}
\newcommand\fig[1]{\Cref{fig:#1}\xspace}
\newcommand\tab[1]{\Cref{tab:#1}\xspace}
\newcommand\tion[1]{\Cref{sect:#1}\xspace}
\newcommand\alg[1]{\Cref{alg:#1}\xspace}
\newcommand\listingref[1]{\cref{line:#1}\xspace}

% --- Shorthands ---
\newcommand{\etal}{\textit{et al.}\xspace}
\newcommand{\ie}{i.e.}
\newcommand{\eg}{e.g.}
\newcommand{\wrt}{\textit{w.r.t.}\xspace}
\newcommand{\etc}{\textit{etc}\xspace}


\newcommand\motivation{{\noindent\textbf{Motivation.}}\xspace}
\newcommand\approach{{\noindent\textbf{Setup.}}\xspace}
\newcommand\observations{{\noindent\textbf{Results.}}\xspace}

\newcommand{\tool}{\textsc{Unicorn}\xspace}
\newcommand{\ourapproach}{\textsc{Unicorn}\xspace}
\newcommand{\ourtool}{$\textsc{Unicorn}_{\textsc{Tool}}$\xspace}
\newcommand{\dd}{$\delta$\textsc{-Debug~}\xspace}
\newcommand{\encore}{\textsc{EnCore}\xspace}
\newcommand{\bugdoc}{\textsc{BugDoc}\xspace}
\newcommand{\cbi}{\textsc{CBI}\xspace}
\newcommand{\TOOL}{\textsc{CADET}\xspace}
\newcommand\nano{\textsc{Nano}\xspace}
\newcommand\txone{\textsc{TX1}\xspace}
\newcommand\txtwo{\textsc{TX2}\xspace}
\newcommand\xavier{\textsc{Xavier}\xspace}
\newcommand\needcitation{\red{[?]}\xspace}
% \newcommand\gpugrowth{\texttt{GPU} \texttt{memory} \texttt{growth}\xspace}
\newcommand\gpugrowth{{GPU} {memory} {growth}\xspace}
% \newcommand\swapmem{\texttt{swap} \texttt{memory}\xspace}
\newcommand\swapmem{{swap} {memory}\xspace}
\newcommand\latency{\texttt{latency}\xspace}
\newcommand{\nfp}{\textsc{nf-fault}\xspace}
\newcommand{\nfps}{\textsc{nf-faults}\xspace}
\newcommand\gpufreq{\texttt{GPU} \texttt{frequency}\xspace}
\newcommand\cpufreq{\texttt{CPU} \texttt{frequency}\xspace}
\newcommand\emcfreq{\texttt{EMC} \texttt{frequency}\xspace}
\newcommand\swappiness{\texttt{Swappiness}\xspace}
\newcommand\contextswitches{\texttt{Context} \texttt{switches}\xspace}
\newcommand\cpucores{\texttt{CPU} \texttt{cores}\xspace}
\newcommand\cachepressure{\texttt{Cache} \texttt{Pressure}\xspace}
\newcommand\schedulerruntime{\texttt{Scheduler} \texttt{run time}\xspace}
\newcommand\cachemisses{\texttt{Cache} \texttt{misses}\xspace}
\newcommand\util{\texttt{CPU} \texttt{utilization}\xspace}
% --- List options ---
\newcommand{\be}{\begin{enumerate}}
\newcommand{\smalleq}{\begin{equation}\small}
\newcommand{\smallereq}{\begin{equation}\footnotesize}
\newcommand{\eeq}{\end{equation}}
\newcommand{\beqml}{\begin{multline}}
\newcommand{\eeqml}{\end{multline}}
\newcommand{\besq}{\begin{enumerate}[leftmargin=*,wide=0pt,topsep=0pt]}
\newcommand{\ee}{\end{enumerate}}
\newcommand{\bi}{\begin{itemize}}
\newcommand{\bicirc}{\begin{itemize}[leftmargin=*]\renewcommand\labelitemi{$\circ$}}
\newcommand{\bisq}{\begin{itemize}[leftmargin=*,wide=0pt,topsep=1pt]}
\newcommand{\ei}{\end{itemize}}
\newcommand{\ourtitle}{\tool: Debugging Hardware non-functional faults Using Causal Inference}
\newcommand{\pdftitle}{\tool: Debugging Hardware non-functional faults Using Causal Inference}
\newcommand{\pdfauthors}{Anonymous Author(s)}

\newcommand\todo[1]{\red{TODO:~#1}\xspace}
\newcommand\rahul[1]{\red{RAHUL:~#1}}
\newcommand\pooyan[1]{\red{POOYAN:~#1}}
\newcommand\rayb[1]{\red{RAYB:~#1}}
\newcommand\mohammad[1]{\red{MOHAMMAD:~#1}}
\newcommand\shahriar[1]{\red{SHAHRIAR:~#1}}


\makeatletter
\edef\textFontName{\fontname\csname
  \f@encoding/\f@family/\f@series/\f@shape/\f@size\endcsname}
\edef\mathFontName{\fontname\textfont0}
\edef\mathLetterFontName{\fontname\textfont1}
\newcommand{\removelatexerror}{\let\@latex@error\@gobble}
% \renewcommand\paragraph{\@startsection{subsubsection*}{3}{\z@}%
%                                      {4pt\@plus 0ex}%
%                                      {0ex \@plus 0ex}%
%                                      {\normalfont\em}}% from \normalsize

% \renewcommand\subsubsubsection*{\@startsection{subsubsection}{3}{\z@}%
%                                     {3pt\@plus 0ex}%
%                                     {0ex \@plus 0ex}%
%                                     {\sc}}% from \normalsize

\makeatother

% --- TIKZ ARROWS ---

\usepackage{tikz}
\usetikzlibrary{arrows.meta}

\newcommand\edgezero{
\begin{tikzpicture}
  \draw[black,  arrows={-Triangle[angle=90:3pt,black,fill=black]}] (0,0.0) -- (0.5,0.0);
\end{tikzpicture}}

\newcommand\edgeone{
\begin{tikzpicture}
  \draw[black,  arrows={-Triangle[angle=90:3pt,black,fill=black]}] (0,0.0) -- (0.5,0.0);
\end{tikzpicture}\xspace}

\newcommand\edgetwo{
\begin{tikzpicture}
  \draw[black,  arrows={Triangle[angle=90:3pt,black,fill=black]-Triangle[angle=90:3pt,black,fill=black]}] (0,0.0) -- (0.5,0.0);   
\end{tikzpicture}\xspace}

\newcommand\edgethree{
\begin{tikzpicture}
  \draw[black,  arrows={Circle[open]-Triangle[angle=90:3pt,black,fill=black]}] (0,0.0) -- (0.5,0.0);   
\end{tikzpicture}\xspace}

\newcommand\edgefour{
\begin{tikzpicture}
  \draw[black,  arrows={Circle[open]-Circle[open]}] (0,0.0) -- (0.5,0.0);   
\end{tikzpicture}\xspace}


% --- MATH MACROS ---
\newcommand\norm[1]{\left\lvert#1\right\lvert}
\usepackage{amsthm}
\DeclareMathOperator*{\argmax}{argmax}
\usepackage{thmtools}
\usepackage{tikz}
\usetikzlibrary{arrows.meta}
\declaretheoremstyle[headfont=\scshape]{schead}
\declaretheoremstyle[headfont=\bf]{bfhead}
\newcommand\independent{\protect\mathpalette{\protect\independenT}{\perp}}
\newcommand\notindependent{\not\independent}
\def\independenT#1#2{\mathrel{\rlap{$#1#2$}\mkern2mu{#1#2}}}
\declaretheorem[style=schead]{definition}
\declaretheorem[style=bfhead]{example}

% --- Algorithm ---
\usepackage[linesnumbered,ruled,vlined, noend]{algorithm2e}
\newcommand\mycommfont[1]{\footnotesize\ttfamily\textcolor{gray70}{#1}}
\SetCommentSty{mycommfont}
\SetAlCapNameFnt{\footnotesize}
\SetAlCapFnt{\footnotesize}
\SetAlgorithmName{Pseudocode}{Pseudocode}{Pseudocode of Algorithms}
\newcommand\algvar[1]{\texttt{#1}\xspace}
\newcommand\alginlinecomment[1]{\grayline{\hfill$\triangleright$~\textsl{#1}}\xspace}
\newcommand\algnestedcomment[1]{\javadocblue{$\triangleright$~\textsl{#1}}\xspace}
\newcommand\algcomment[1]{\javadocblue{$\triangleright$~\text{#1}\xspace}}

% --- Material style Result Box ---
\usepackage[framemethod=tikz]{mdframed}
\usetikzlibrary{shadows}
\usepackage{graphics}
\newmdenv[
    tikzsetting= {fill=gray05!0},
    skipabove=0.33em,
    skipbelow=0.33em,
    linewidth=1pt,
    innerleftmargin=4pt,
    innerrightmargin=4pt,
    innertopmargin=2pt,
    innerbottommargin=2pt,
    linecolor=gray85,
    roundcorner=2pt, 
    shadow=true,
    shadowsize=4pt,
    shadowcolor=black
]{myshadowbox}
\newmdenv[
    tikzsetting= {fill=gray05!0},
    skipabove=0.33em,
    skipbelow=0.33em,
    linewidth=1.25pt,
    innerleftmargin=4pt,
    innerrightmargin=4pt,
    innertopmargin=2pt,
    innerbottommargin=2pt,
    linecolor=gray85,
    roundcorner=3pt, 
    shadow=false,
    shadowsize=3pt,
    shadowcolor=black
]{mygoalbox}
\usepackage{tikz}
\newcommand*\circled[1]{\tikz[baseline=(char.base)]{
            \node[shape=circle,draw,inner sep=0.5pt] (char) {\small #1};}}

\newenvironment{result}
{\begin{myshadowbox}
  \besq
  \item[\faHandORight]
    % \item[]
  % \item[{\footnotesize \faPlay}]
  }
{\ee\end{myshadowbox}}

\newenvironment{goal}
{\vspace{0.2em}
\begin{mygoalbox}
  \besq
    % \item[\faHandORight]
    \item[{\footnotesize \faPlay}]
}
{
  \ee
  \end{mygoalbox}
}
%\renewcommand{\Bbbk}{\mathbb{k}}
%\setlength{\textfloatsep}{10pt plus 1.0pt minus 2.0pt}

\newcommand\given[1][]{\:#1\vert\:}