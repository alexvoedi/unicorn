\section{Scalability of \tool}

To examine the scalability of \tool, we evaluate \tool using the speedtest workload of SQLite database to resolve the latency faults due to misconfigurations. SQLite database workloads have a higher number of configurable options compared to image, NLP, and speech workloads. SQLite workloads exhibit unique compute, memory, data reuse characteristics, and higher memory accesses. As a result, SQLite exposes new system design opportunities to enable efficient inference and many complex interactions between software options. We evaluate \tool on the following two scenarios:

\noindent\paragraph{Increasing number of configuration options.} In the first scenario, we perform a sensitivity analysis of \tool by increasing the size of the configuration space. We start with 47 configuration options which we manually select based on important hardware, kernel configuration options, and system events recommended in performance blogs. Then we increase the number of configuration options to 330 by including 206 \texttt{sysctl} kernel configuration options that can be tuned without rebooting the system. Finally, we extend the number of configuration options to 530 by including all the system events tracked by \textsc{iperf}. In each case, we track the time required to discover the causal structure, evaluate the counterfactual queries and resolve a latency fault.  

\noindent\paragraph{Increasing number of configuration options' values.} \tool recommends a fix by computing individual treatment effect to evaluate counterfactual queries. The number of these counterfactual queries depends on the number of values taken by a configuration option. If a configuration option can be set to a large number of values, the number of counterfactual queries needed to be evaluated also increases. To determine the behavior of \tool in such scenarios, we evaluate \tool by increasing the number of values the hardware and kernel configuration options can be set at while resolving the latency faults in the SQLite speedtest workload. 
